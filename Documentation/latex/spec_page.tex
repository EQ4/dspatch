\subsection*{1. The Circuit Concept\-:}

A circuit is comprised of a collection of interconnected components. Each component has 2 signal buses, on one end of the component there are input pins (input bus), and on the other end there are output pins (output bus). Components within the circuit are connected to each other via wires. Each wire carries a signal from one component's output pin to another component's input pin. A circuit can also comprise of interconnected circuits (E.\-g. I\-Cs). In this case a circuit acts as a component within another circuit.





\subsection*{2. The D\-S\-Patch Circuit System\-:}

{\bfseries 2.\-1 Structure\-:}

The nouns above are the classes we require in order to model our circuit in code. Each component will contain an array of input wires. Each wire contains references to the source component, the source output signal, and the destination input signal. The signal bus class will contain an array of signals, and lastly, the circuit class is derived from component and will contain an array of internal components.

{\bfseries 2.\-2 Behavior\-:}

{\bfseries 2.\-2.\-1 Component\-:}

The component class will have a Tick() method responsible for acquiring its next set of inputs from its input wires and populating the component's input bus. To insure that these inputs are up-\/to-\/date, the dependent component first calls all of its input components' Tick() methods -\/hence recursively called in all components going backward through the circuit. The acquired input bus is then passed into a pure virtual method\-: Process() -\/it is the responsibility of the (derived) component creator to implement this virtual function. The Process() method has 2 input parameters\-: the input bus and the output bus. This method's purpose is to pull its required inputs out of the input bus, process these inputs, and populate the output bus with the results. These resultant outputs in the output bus are then acquired by dependent components via their Tick() functions.

Component input and output buses (signal count, string I\-Ds etc.) will be configurable by derived component classes via protected base methods, whilst input wires to a component will be routable via public component class methods.

The component class will also have a Reset() method. For optimization as well as to avoid feedback deadlocks, a component needs to be aware of whether or not it has already ticked during a circuit traversal so that if called to Tick() again, it can ignore the call. The Reset() method informs the component that the last circuit traversal has completed and hence can execute the next Tick() request.

When a component is instantiated within a circuit, a unique string I\-D is required. This name is used to identify the component in the circuit's component collection.

{\bfseries 1.\-2.\-2 Circuit\-:}

In order to satisfy the statement above (\char`\"{}circuit acts as a component\char`\"{}), the circuit class is derived from the component class. This means that the circuit class has both Tick() and Process() methods. The Tick() method will execute as normal, acquiring inputs for the circuit to process. This allows us to expose the I\-O we require for internal components via the circuit's input and output buses. Circuit I\-O-\/to-\/component wires, and component-\/to-\/component wires, will be publicly routable via circuit class methods. The circuit class' virtual Process() method is implemented as such\-:

All internal components are Reset()ed. Inputs from the circuit's input bus are copied into their respective internal component input buses. All internal components are Tick()ed. The circuit output bus is populated with the respective internal component outputs.

All actions in respect to the circuit and the components within the circuit will be made available via public functions in the circuit class. The circuit object user will be able to add/remove components, connect/disconnect wires, set/get circuit I\-O, set/get individual component I\-O, and acquire a handle to any component within the circuit. This means only ever requiring one object to perform all circuit actions.

{\bfseries 1.\-2.\-3 Signal\-:}

When it comes to transferring signals between components we require the same level of abstraction for the data being moved around\-:

The base component class needs to supply its child class with any number of inputs and outputs via the virtual Process() method. These inputs and outputs may also need to be of different types. This requires a generic way of containing variables of different types in a single collection -\/the signal bus.

The signal class will hold a variable that can be dynamically typed at run-\/time, which I've named \char`\"{}run-\/type\char`\"{}. The run-\/type and signal classes make use of template methods to allow object users to set and get the contained variable as any type they wish. The run-\/type (and hence, a signal) has the ability to change type at any point during program execution -\/this can be useful for inputs that can accept a number of different types of data (E.\-g. Varying sample size in an audio buffer\-: array of byte / int / float)

From the Process() method, a derived component can get and set the signals it requires of the provided signal buses via public methods. As the component creator is responsible for configuring the component's I\-O buses, the types held within those buses can be assumed, and hence, read and written to accordingly. Built-\/in typecasting and error checking prevents critical run-\/time errors from occurring when signal types are mismatched.

When a signal is instantiated, an optional unique string I\-D can be assigned. This name can be used to identify the signal in collections (E.\-g. a signal bus). Alternatively, signals are referred to via index. 